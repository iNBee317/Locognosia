\documentclass[man, natbib]{apa6}
\usepackage[utf8]{inputenc}


\title{Locognosia}
\shorttitle{Locognosia}

\author{Nathan A. Baune}

\affiliation{University of Missouri-Columbia}

\leftheader{Baune}

\abstract{}

\keywords{}

\usepackage{Sweave}
\begin{document}


\maketitle 


\section{Methods}
\subsection{Participants}
Four unilateral hand replant patients (mean age $\pm$ SD = 55.5 $\pm$ 5.07, all right handed), three unilateral hand transplant recipients (age 41.8 $\pm$ 5.72, 1 left handed), and fourteen controls (age 53 $\pm$ 11.09, all right handed) were recruited from ?. All participants were male. All participants provided informed consent.  See figure 1 for handedness and affected hand by participant. For participant DR, data was collected on three separate occasions. Data was collected once for all other participants.

\subsection{Procedure and Materials} 
Using a method designed by \cite{noordenbos1972sensory} participants were tested on their ability to localize tactile stimuli (locognosia) applied to the palmar surface of their hands. Participants were seated at a table with a hand resting palm side up on the table surface. With the subject adorning red tinted goggles and looking away the investigator marked 15 points on the palmar surface with a pink pen according to a predetermined schematic, see fig. 2. The participants would continue to look away as the investigator applied stimuli to a single point using a 6.10 gram Semmes-Weinstein monofilament. Upon a verbal cue the participants redirected their gaze and marked the perceived location of the stimuli with an orange pen held in their opposite hand. The participant then looked away as the investigator measured the distance between the target dot and the response to the nearest 1 mm. This was repeated for all 15 points on each hand. Neither the pink target dots or the orange response dots were visible to the participants as long as the red goggles were in place. The goggles were left on for the entire extent of the experiment.
     

\section{Results}

Table \ref{tab:ComplexTable} contains some sample data. Our

statistical prowess in analyzing these data is unmatched.

\begin{table}[htbp]

\vspace*{2em}

\begin{threeparttable}

\caption{Handedness}

\label{tab:ComplexTable}

\begin{tabular}{lrrrrr} \toprule

Group & Participant & Age& Gender& Hand& \\ \cmidrule(r){5-6}

& & & & Dominant & Affected\\ \midrule

Replant & CH(WH) & 60&Male& Right & Left \\

& JS & 49&Male&Right & Right\\ 

& PP & 61&Female&Right & Right\\

& RW & 59&Male&Right & Left \\ \midrule

Transplant & DR & 37&Male&Right & Left\\

& GF & 46&Male&Right & Left\\

& MS &49 &Male&Left & Left\\ \midrule

\end{tabular}

\begin{tablenotes}[para,flushleft]

{\small

%\textit{Note.} All data are approximate.

%\tabfnt{a}Categorical may be onset.

%\tabfnt{b}Categorical may also be coda.

%\tabfnt{*}\textit{p} < .05.

%\tabfnt{**}\textit{p} < .01.

}

\end{tablenotes}

\end{threeparttable}

\end{table}

\section{Discussion}

\bibliography{nbref}
\end{document}
